\section{Materiais e Métodos}
\label{sec:materiais}

Descrever aqui o que foi feito durante o projeto e colocar o fluxograma.
Todo o projeto foi desenvolvido e documentado no Github (colocar o link)

\subsection{Correlação com os múculos da Face}
Aqui foi tentado fazer algumas correlações com os músculos faciais, contudo, ao que foi descoberto no item ?? sobre a analise de exprsssões em vídeos e a alta complexidade dos músculos, foi abandonada esta estratégia e optado por fazer a redução dos dados para otimizar a análise utilizando agrupamento por KMeans descrito no item ??.

\subsection{Experimentos}
\subsubsection{Experimento 1 - Analise dos rostos e coleta dos dados de LandMarks}
O objetivo é coletar os dados para que possam ser analisados e correlacionados posteriormente.

Foi analisada cada imagem de cada pessoa em cada uma das expressões e o resultado será salvo na pasta processed da seguinte forma.
\begin{verbatim}
.
├── /processed
|   |
|   ├── [id_expressão]
|   |   |
|   |   ├── mean_[id_expressão].png (Mascara com triangulos fantas da expressão com fundo branco)
|   |   ├── [id_supervisionado]
|   |   |   ├── [id_pessoa]
|   |   |   |   |
|   |   |   |   ├── [id_foto].csv (vai conter as posições xy dos landmarks da foto)
|   |   |   |   ├── [id_foto].jpg (Mascara de triangulos da pessoa com fundo preto pra cada foto)
|   |   |   |   ├── mean_[id_pessoa].jpg (Mascara de triangulos fantasma da pessoa com fundo branco)

\end{verbatim}

\paragraph{Descrição dos Arquivos}
\begin{itemize}
    \item{analyser.py}
    \subitem{Anda de pasta em pasta do dataset e salva os .csv em processed na pasta raiz do projeto}

    \item{draw\_triangles.py}
    \subitem{Anda pelas pastas de processed e, a partir dos csv, gera as mascaras e as salva em .jpg no mesmo diretório}

    \item{error\_verifier.py}
    \subitem{Verifica e exclui as inconsistencias encontradas no dataset, como pastas com numeros equivocados de imagens, e casos nos quais faces não foram reconhecidas}

    \item{face\_mesh.py}
    \subitem{Uma classe para extrair os pontos da face}

    \item{face\_adjustments.py}
    \subitem{Faz as transformações na imagem a partir dos pontos extraidos, de acordo com as definições abaixo}

\end{itemize}

\paragraph{Procedimentos}\mbox{}\\



Para melhor analisar as imagens posteriormente um padrão será adotado, e ele será o seguinte:
\begin{itemize}
    \item{Todas as imagens serão rotacionadas para que os olhos fiquem sempre na mesma linha (alignEyes)}
    \item{O ponto 10 é deixado sempre na altura 25 e no centro da imagem, com a distancia entre o ponto 10 e o ponto 152 sendo de 1500px (alignFace)}
    \subitem{Colocar aqui uma imagem com as linhas mostrando o tamanho do rosto sendo 1500px}
    \item{A imagem será recortada em um retangulo, deixando apenas o rosto centralizado, com margem de 25px.}
    \item{A imagem será deixada com 1900px X 2500px sem redimensionar, apenas colocando bordas pretas nas laterais e em baixo para completar o tamanho}
    \item{somente então será salva com os dados relativos a imagem no fim deste processo}
\end{itemize}
Colocar aqui o bloco do experimento 1 do fluxograma


\paragraph{Remoções do Dataset Neste Experimento}\mbox{}\\
\subparagraph{Motivo: Número incorreto de fotos em uma pasta do usuário}\mbox{}\\
nos casos em que foi detectado que uma das pastas do usuário continha um numero diferente de 3 fotos, todas as pastas dele foram removidas da analise. Os usuários que esta regra foi aplicada são os que seguem, com as seguintes observações:
\begin{itemize}
    \item{00541 - 4 imagens em 00/00}
    \item{02069 - 11 imagens em 00/00}
    \item{05669 - 7 imagens em 00/00}
    \item{06644 - 2 imagens em 00/00}
\end{itemize}

\paragraph{Exemplo de Máscara gerada com os Landmarks obtidos}\mbox{}\\
colocar a imagem aqui da mascara com o fundo preto

\paragraph{Vídeo das Faces e Máscaras}\mbox{}\\
Fui obrigado a colocar o video no drive pois ele acabou ficando ridiculamente pesado (cerca de 4,5gb) O codec utilizado não é suportado pelo reprodutor do windows, abra o video com o VLC que irá ser lido normalmente. O as imagens utilizadas no video tem 1/3 de seu tamanho original, para reduzir o tamanho do mesmo. Deixando-o com a resolução de 1666x633 px
(colocar o link pro vídeo ou um print dele sla)
No video as informações estão dispostas da seguinte forma:

\begin{itemize}
    \item{Na esquerda está o rosto cortado da pessoa}
    \item{Na direita a máscara extraida deste rosto}
    \item{no meio estão o tipo/expressão (ids) e o id da pessoa, respectivamente}
    \item{a cada troca de pasta (tipo ou expressão) é mostrado por 2s uma imagem do novo diretorio com os ids (tipo/expressão)}
\end{itemize}


\subsubsection{Experimento 2 - Processamentos dos Dados Obtidos no Experimento 1}
O objetivo aqui é tentar encontrar um padrão para os dados obtidos no experimento 1, e fazer um processamento dos dados para que possam ser utilizados no experimento 3.

\paragraph{Primeira Abordagem - Todos os Pontos (Foi abandonada)}\mbox{}\\
\subparagraph{Parte 1}\mbox{}\\
Para cada uma das máscaras de um usuário, obter as distâncias de cada ponto desta máscara para sua mascára neutra não supervisionada (00/00). Com esses dados, também gerar distâncias médias para a neutra NS de todos os usuários

\subparagraph{Parte 2}\mbox{}\\
A partir dos dados gerados na parte 1, que são MUITOS, reduzílos a alguns números que, em teoria, representam a mascara daquela pessoa. Os resultados que serão coletados são os que seguem:
\begin{itemize}
    \item{Média de diferença entre as distancias da máscara em análise para todas as distancias da máscara neutra NS média}
    \subitem{Isso significa que eu tiro a distancia do Ponto X para todos os outros pontos da máscara em análise e faço o mesmo para a másca de referencia, que é a Neutra NS. Tendo esses 2 .csv de 468x486 linhas cada, faço a diferença entre cada uma das relações e coloco em um .csv com 468x486 (isso não é uma matriz, é uma multiplicação msm) linhas}
    \item{Média de diferença entre as distancias da máscara em análise para todas as distancias da máscara neutra NS média, agrupado por músculos (Ver MuscleCorrelation)}
    \subitem{Faço a mesma coisa que no item acima, contudo a média será feita por músculos}
    \item{Média de diferença entre as distancias da máscara em análise para todas as distancias da máscara neutra NS média, agrupado por GRUPOS musculares}
    \subitem{Faço a mesma coisa que no item acima, contudo a média será feita por grupos musculares}
\end{itemize}

Todos os dados citados acima serão salvos em .cvs's, de acordo com a relação dos arquivos gerados
\begin{verbatim}
.
├── /processed
|
│   ├── /pattern_analysis
│   |   |
│   |   ├── /[id_pessoa]
│   |   |   |
│   |   |   ├── exp-tp_00-00.csv (A diferença entre cada ponto das expressões para o neutro NS)
│   |   |   ├── mean-dists_allp-allp.csv (A media de distancia entre cada ponto das expressões para o neutro NS)
│   |   |   ├── mean-diff-dists_allp-allp.csv (A diferença entre cada ponto das expressões para o neutro NS média)
\end{verbatim}
Contudo como foi dito no item ?? foi abandonada a questão de agrupamento de pontos por músculos epor conta disso, foi adotada a abordagem descrita no próximo item.

\paragraph{Segunda abordagem - Pontos com alta variação em relação à face neutra}\mbox{}\\
Neste experimento apenas serão analisados os pontos e suas contrapartes na expressão neutra. A primeira parte foi calcular a média de distancia de cada ponto de todas as 16 máscaras para a neutra e então selecionei todos os pontos que a distancia é superior a média da máscara em todas as máscaras. 77 pontos foram selecionados
\newline
(não sei se é interessante colocar os graficos de distancias de cada uma das mascaras)
(e eu acho que essa imagem é um resultado, logo nao deveria estar nessa section deveria estar na proxima tb)

\includegraphics[width=0.5\textwidth]{images/pav_mask.png}

Com estes 77 pontos em mão, para cada máscara foi calculada a média deles, criando o arquivo masks\_means\_pav.csv
pav = points above average

\paragraph{Remoções do dataset}
\begin{itemize}
    \item {Usuário 06644: Não fez todas as capturas}
\end{itemize}



\subsubsection{Experimento 3 - Visualisação dos Dados}
\paragraph{Gráfcos de PseudoClustering}\mbox{}\\
\paragraph{Analise da distribuição das distancias da segunda abordagem do experimento 2}\mbox{}\\
\paragraph{Cluster com Kmeans dos Pontos acima da Média (PAV)}\mbox{}\\
Como foram selecionados 77 pontos, houve uma sobrecarga de dados para fazer a analise. Então foi utilizada esta técnica de clusterização para reduzir a quantidade de dados. Os 77 pontos foram dividos em 6 grupos, mostrados na imagem abaixo
(tb n sei se é aqui que eu dvo colocar a img)
A relação ponto-grupo esta salvo no arquivo points\_above\_average.csv Para a execução dos testes na secção 04 será utilizada a média de cada de distância de cada ponto

\subparagraph{Expansão dos grupos}\mbox{}\\
Dois novos grupos serão adicionados para analise arbitrariamente para ser feita uma visualização da relação do tamanho do sorriso com a contração dos olhos, para verificar se o sorriso é, de fato verdadeiro. Esta nova relação de grupos está no arquivo points\_above\_average\_expanded.csv

\paragraph{Remoções do Dataset}\mbox{}\\
\begin{itemize}
    \item {02935: Não respondeu aos questionários}
\end{itemize}

\subsubsection{Experimento 4 - Cálculo de Correlações}
aqui eu vou escrever depois pq eu não anotei no github...
\subsubsection{Experimento 5 - Testes de Hipóteses}
\paragraph{pessoa com score mais alto em depressão tende a ter uma media maior das distancias da face neutra pra neutra media dos participantes}
isso aqui serviu p testar se os sorrisos eram verdadeiros
explicar que eu vou fazer o teste com o score bdi e o kmeans expandido pra pegar a contração do olho
