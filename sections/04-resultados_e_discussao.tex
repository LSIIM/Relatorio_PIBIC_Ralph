\section{Resultados e Discussão}
\label{sec:resultados}

\subsection{Máscaras}
\subsubsection{Máscaras médas por expressão}\mbox{}\\
\subsubsection{Diferenças entre Supervisionado e Não SUpervisionado}\mbox{}\\
Notou-se que:
\begin{itemize}
    \item Na 05 (Alegria) foi mais comum rostos mais expressivos e com sorrisos maiores na captura \emph{não supervisionada} do que na supervisionada.
    \item Na 07 (Surpresa) aconteceu o oposto da 05, pois foi mais comum bocas com uma amplitude de abertura maior nas capturas \emph{supervisionadas} do que nas não supervisionadas
\end{itemize}

\subsubsection{Comparações entre máscaras médias supervisionadas e não supervisionadas}\mbox{}\\

A coluna da esquerda representa as mascaras não supervisionadas e a da direita as supervisionadas. A cada Bloco de imagem são apresentadas 4 imagens sendo duas máscaras "fantasmas" e duas de pontos médios.
A máscara fantasma apresenta os vestígios de todas as mascaras analisadas simultaneamente, reforçando os pixels onde as linhas mais aparecem.
A máscara de pontos médios apenas faz uma média das posições x,y de cada um dos landmarks da expressão e desenha uma máscara a partir disso.

COLOCAR AS IMAGENS AQUI


\subsubsection{Distância entre os pontos da expressão Neutra Não Supervisionada}\mbox{}\\

Aqui foram geradas visualizações das máscaras obtendo a distancia entre os landmarks da máscara média de uma dada expressão para a média da Neutra não supervisionada. E para gerar o heatmap foi pintado um triângulo com o valor médio dos 3 vértices que o compõem.

COLOCAR AS IMAGENS AQUI DA ESCALA DINAMICA (ACHO QUE AQUELA COR TA ÓTIMA)

\subsubsection{Grupos Faciais}\mbox{}\\
Como foi dito no item ?? foram separados 77 pontos cujas distâncias até seu correspondente na neutra NS são maiores que a média do restante de sua máscara.
Para obter estes pontos foram analisados os seguintes dados aqui representados em gráficos:

COLOCAR OS GRÁFICOS AQUI

E então foram separados nos seguintes pontos destacados em vermelho

COLOCAR A IMAGEM COM OS PONTOS DESTACADOS AQUI (NÃO COLOCAR ELA NO 3)

E conforme foi observado pelo bolsista, a quantidade de dados ainda era absurdamente grande, portanto foi necessário reduzir a quantidade de dados para aumentar a velocidade de processamento. Para tal foi utilizado o método de K-means para reduzir a quantidade de dados. Gerando os seguintes grupos de pontos

COLOCAR A IMAGEM DO KMEANS PADRÃO

\subsection{Correlações Mais Fortes}
Conforme foi discutido no item ?? os testes de correlação foram feitos utilizando os grupos faciais obtidos pelo método de clusterização Kmeans, e correlacionando-os com os scores dos questionários psiquiátricos descritos no item ??. (é seria legal descrever eles e colocar eles no apendice talvez, ou anexo sla)

Algumas correlações fortes foram encontradas, como a Beck Anxiety Inventory VERSUS Obsessive-Compulsive Inventory, contudo como este tipo de correlação entre scores dos questionários ja foi amplamente discutido na literatura, não foi dispendido tempo para analisá-los.

As melhores correlações obtidas foram:


\begin{itemize}
    \item Grupo 02 da Alegria não supervisionada VERSUS Adult Self-Report Scale (Hiperatividade)
    \begin{itemize}
        \item spearmanr
            \subitem pvalue = 0.0001702527593707
            \subitem rvalue = 0.2122881902011349
        \item pearsonr
            \subitem pvalue = 6.888895152022905e-05
            \subitem rvalue = 0.2244339895379587
    \end{itemize}
\item Grupo 02 da Alegria supervisionada VERSUS Adult Self-Report Scale (Hiperatividade)
    \begin{itemize}
        \item spearmanr
            \subitem pvalue = 0.0003732745538004
            \subitem rvalue = 0.2011662537681749
        \item pearsonr
            \subitem pvalue = 0.0043378892933926
            \subitem rvalue = 0.1618568566694145
    \end{itemize}
\end{itemize}


\subsection{Problema de Aquisição}

\subsubsection{Testes}\mbox{}\\

Para avaliar a hipótese de que um sorriso verdadeiro contrai os olhos e que pessoas com score alto no DBI não expressam sorrisos tão fortes, foi criado mais 2 grupos faciais arbitrários na mascara, apresentados abaixo:

COLOCAR KMEANS EXPANDIDO AQUI

Com isso, seria possível verificar a média de distância do grupo 5 com o grupo 6 e do grupo 1 com o grupo 7. Assim, seria esperado encontrar dados como o exemplo abaixo:

COLOCAR O GRAFICO esperado

Contudo o que foi encontrado foi:

COLOCAR UNS PRINTS E LINKS PROS HTML INTERATIVOS

E observando os gráficos nota-se que não condiz com o que era esperado, portanto pode-se inferir que a coleta criou este erro ou a premissa da hipótese era falsa.
Além disso, para a analise de expressões faciais é necessário que hajam vídeos da expressão surgindo e se dissipando (buscar referencia disso)